% !TEX encoding = UTF-8 Unicode
%
% 2025-07-04 産技高専品川情報・卒研中間発表レジメサンプル
% 2025-12-01 産技高専品川情報・卒研発表レジメサンプル
%
\documentclass[twocolumn,dvipdfmx,a4j]{jarticle}

\usepackage{is-resume}% ISレジメスタイルファイル読み込み

\course{情報システム工学コース}
\event{2025年度卒業研究発表会}

\title{題名例---京浜運河のスループットと花火の相関}
%\title{題名例---京浜運河のスループットと花火の相関\\
%--- サブタイトルがある場合 ---
%}
\author{5499 品川 太郎}
\supervisor{坂東 小次郎 \normalsize{}教授}

\begin{document}

\maketitle

\section{背景と目的}    

ここは本文の例である。
\LaTeX で段落を分ける際には空行を挿入する。
\verb+\\+を使っている者がいるが、それは強制改行である。
段落の字下げなどが効かないので使わないこと。
余白調整の\verb+\vspace+や\verb+\hspace+も極力使わない方が良い。
○○○○○○○○○○○○○○○○○○○○○○○○○○○○○○○○○○○○○○○○○○○○ \cite{文献01}。
○○○○○○○○○○○○○○○○○○○○○○○○○○○○○○○○○○○○○○○○○○○○○○○○。
○○○○○○○○○○○○○○○○○○○○○○○○○○○○○○○○○○○○○○○○○○○○○○○○。
○○○○○○○○○○○○○○○○○○○○○○○○○○○○○○○○○○○○○○○○○○○○○○○○。
○○○○○○○○○○○○○○○○○○○○○○○○○○○○○○○○○○○○○○○○○○○○○○○○。
○○○○○○○○○○○○○○○○○○○○○○○○○○○○○○○○○○○○○○○○○○○○○○○○。

○○○○○○○○○○○○○○○○○○○○○○○○○○○○○○○○○○○○○○○○○○○○○○○○。
○○○○○○○○○○○○○○○○○○○○○○○○○○○○○○○○○○○○○○○○○○○○○○○○。
○○○○○○○○○○○○○○○○○○○○○○○○○○○○○○○○○○○○○○○○○○○○○○○○。

○○○○○○○○○○○○○○○○○○○○○○○○○○○○○○○○○○○○○○○○○○○○○○○○。
○○○○○○○○○○○○○○○○○○○○○○○○○○○○○○○○○○○○○○○○○○○○○○○○。
○○○○○○○○○○○○○○○○○○○○○○○○○○○○○○○○○○○○○○○○○○○○○○○○。
○○○○○○○○○○○○○○○○○○○○○○○○○○○○○○○○○○○○○○○○○○○○○○○○。

\section{○○}

○○○○○○○○○○○○○○○○○○○○○○○○○○○○○○○○○○○○○○○○○○○○○○○○。
○○○○○○○○○○○○○○○○○○○○○○○○○○○○○○○○○○○○○○○○○○○○○○○○ \cite{文献02}。
○○○○○○○○○○○○○○○○○○○○○○○○○○○○○○○○○○○○○○○○○○○○○○○○。
○○○○○○○○○○○○○○○○○○○○○○○○○○○○○○○○○○○○○○○○○○○○○○○○。
○○○○○○○○○○○○○○○○○○○○○○○○○○○○○○○○○○○○○○○○○○○○○○○○。
○○○○○○○○○○○○○○○○○○○○○○○○○○○○○○○○○○○○○○○○○○○○○○○○。

\subsection{□□□}

○○○○○○○○○○○○○○○○○○○○○○○○○○○○○○○○○○○○○○○○○○○○○○○○。
○○○○○○○○○○○○○○○○○○○○○○○○○○○○○○○○○○○○○○○○○○○○○○○○。
○○○○○○○○○○○○○○○○○○○○○○○○○○○○○○○○○○○○○○○○○○○○○○○○。
○○○○○○○○○○○○○○○○○○○○○○○○○○○○○○○○○○○○○○○○○○○○○○○○。
○○○○○○○○○○○○○○○○○○○○○○○○○○○○○○○○○○○○○○○○○○○○○○○○。
○○○○○○○○○○○○○○○○○○○○○○○○○○○○○○○○○○○○○○○○○○○○○○○○。

○○○○○○○○○○○○○○○○○○○○○○○○○○○○○○○○○○○○○○○○○○○○○○○○。
○○○○○○○○○○○○○○○○○○○○○○○○○○○○○○○○○○○○○○○○○○○○○○○○。
○○○○○○○○○○○○○○○○○○○○○○○○○○○○○○○○○○○○○○○○○○○○○○○○。
○○○○○○○○○○○○○○○○○○○○○○○○○○○○○○○○○○○○○○○○○○○○○○○○。
○○○○○○○○○○○○○○○○○○○○○○○○○○○○○○○○○○○○○○○○○○○○○○○○。
○○○○○○○○○○○○○○○○○○○○○○○○○○○○○○○○○○○○○○○○○○○○○○○○。

\begin{itemize}
\item ○○○○○○○○○○○○○○○○○○○○○○○○○○○○○○○○○○○○○○○○○。
\item ○○○○○○○○○○○○○○○○○○○○○○○○○○○○○○○○○○○○○○○○○。
\item ○○○○○○○○○○○○○○○○○○○○○○○○○○○○○○○○○○○○○○○○○。
\end{itemize}

○○○○○○○○○○○○○○○○○○○○○○○○○○○○○○○○○○○○○○○○○○○○○○○○。
○○○○○○○○○○○○○○○○○○○○○○○○○○○○○○○○○○○○○○○○○○○○○○○○。
○○○○○○○○○○○○○○○○○○○○○○○○○○○○○○○○○○○○○○○○○○○○○○○○。
○○○○○○○○○○○○○○○○○○○○○○○○○○○○○○○○○○○○○○○○○○○○○○○○。
○○○○○○○○○○○○○○○○○○○○○○○○○○○○○○○○○○○○○○○○○○○○○○○○。
○○○○○○○○○○○○○○○○○○○○○○○○○○○○○○○○○○○○○○○○○○○○○○○○。

\begin{figure}[tb]
\centering
%\includegraphics[width=50mm,clip]{fig01.png}
\fbox{%
  \vrule width 0pt height 10zw depth 0pt
  \vrule width 20zw height 0pt depth 0pt
}
\caption{図の例---全体図}\label{fig:overview}
% 図のキャプションは下
\end{figure}

\begin{table*}[tb]
\centering
% 表のキャプションは上
\caption{表の例---機材仕様}\label{tbl:specs}
\begin{tabular}{c|l|l|p{9zw}|l|l}
\hline
  & \multicolumn{4}{c|}{\bf 内蔵}
    & \multicolumn{1}{c}{\bf 外付} \\ \cline{2-6} 
  & \multicolumn{1}{c|}{\bf CPU}
    & \multicolumn{1}{c|}{\bf メモリ}
    & \multicolumn{1}{c|}{\bf ストレージ}
    & \multicolumn{1}{c|}{\bf ネットワーク}
    & \multicolumn{1}{c}{\bf ネットワーク} \\
\hline
A & Alpha 3GHz & 256MB & 50GB SCSI-160 & FastEthernet & GigabitEthernet \\
\hline
B & Pentium 4GHz & 128MB
    & とても長い文字列はその欄のフォーマットを p にするか→
    & \begin{tabular}[t]{@{}l@{}} 
      内部でさらに\\
      tabular環境で\\
      折りたたむ
      \end{tabular}
    & 10GBASE-T \\
\hline
\end{tabular}
\end{table*}

図\ref{fig:overview}は、○○○○○○○○○○○○○○○○○○○○○○○○○○○○○○○○○○○○○○○○○○○○○○○○ 。
図は figure環境を、2段抜きの図を挿入する場合は figure*環境を使う。
○○○○○○○○○○○○○○○○○○○○○○○○○○○○○○○○○○○○○○○○○○○○○○○○。
○○○○○○○○○○○○○○○○○○○○○○○○○○○○○○○○○○○○○○○○○○○○○○○○。


\section{○○}

○○○○○○○○○○○○○○○○○○○○○○○○○○○○○○○○○○○○○○○○○○○○○○○○。
○○○○○○○○○○○○○○○○○○○○○○○○○○○○○○○○○○○○○○○○○○○○○○○○を表\ref{tbl:specs}に示す。
表は table環境を、2段抜きの表を挿入する場合は table*環境を使う。
○○○○○○○○○○○○○○○○○○○○○○○○○○○○○○○○○○○○○○○○○○○○○○○○。

○○○○○○○○○○○○○○○○○○○○○○○○○○○○○○○○○○○○○○○○○○○○○○○○。
○○○○○○○○○○○○○○○○○○○○○○○○○○○○○○○○○○○○○○○○○○○○○○○○。
○○○○○○○○○○○○○○○○○○○○○○○○○○○○○○○○○○○○○○○○○○○○○○○○。
○○○○○○○○○○○○○○○○○○○○○○○○○○○○○○○○○○○○○○○○○○○○○○○○。
○○○○○○○○○○○○○○○○○○○○○○○○○○○○○○○○○○○○○○○○○○○○○○○○。
○○○○○○○○○○○○○○○○○○○○○○○○○○○○○○○○○○○○○○○○○○○○○○○○。
○○○○○○○○○○○○○○○○○○○○○○○○○○○○○○○○○○○○○○○○○○○○○○○○。
○○○○○○○○○○○○○○○○○○○○○○○○○○○○○○○○○○○○○○○○○○○○○○○○。
○○○○○○○○○○○○○○○○○○○○○○○○○○○○○○○○○○○○○○○○○○○○○○○○。

○○○○○○○○○○○○○○○○○○○○○○○○○○○○○○○○○○○○○○○○○○○○○○○○。
○○○○○○○○○○○○○○○○○○○○○○○○○○○○○○○○○○○○○○○○○○○○○○○○。
○○○○○○○○○○○○○○○○○○○○○○○○○○○○○○○○○○○○○○○○○○○○○○○○。
○○○○○○○○○○○○○○○○○○○○○○○○○○○○○○○○○○○○○○○○○○○○○○○○。
○○○○○○○○○○○○○○○○○○○○○○○○○○○○○○○○○○○○○○○○○○○○○○○○。
○○○○○○○○○○○○○○○○○○○○○○○○○○○○○○○○○○○○○○○○○○○○○○○○。

\section{○○}

○○○○○○○○○○○○○○○○○○○○○○○○○○○○○○○○○○○○○○○○○○○○○○○○。
○○○○○○○○○○○○○○○○○○○○○○○○○○○○○○○○○○○○○○○○○○○○○○○○。
○○○○○○○○○○○○○○○○○○○○○○○○○○○○○○○○○○○○○○○○○○○○○○○○。
○○○○○○○○○○○○○○○○○○○○○○○○○○○○○○○○○○○○○○○○○○○○○○○○。

○○○○○○○○○○○○○○○○○○○○○○○○○○○○○○○○○○○○○○○○○○○○○○○○。
○○○○○○○○○○○○○○○○○○○○○○○○○○○○○○○○○○○○○○○○○○○○○○○○。
○○○○○○○○○○○○○○○○○○○○○○○○○○○○○○○○○○○○○○○○○○○○○○○○○。
○○○○○○○○○○○○○○○○○○○○○○○○○○○○○○○○○○○○○○○○○○○○○○○○。
○○○○○○○○○○○○○○○○○○○○○○○○○○○○○○○○○○○○○○○○○○○○○○○○。
○○○○○○○○○○○○○○○○○○○○○○○○○○○○○○○○○○○○○○○○○○○○○○○○。


\section{○○}

%「はじめに」と「おわりに」、「序論」と「結論」など対比を気にすること。
○○○○○○○○○○○○○○○○○○○○○○○○○○○○○○○○○○○○○○○○○○○○○○○○。
○○○○○○○○○○○○○○○○○○○○○○○○○○○○○○○○○○○○○○○○○○○○○○○○。
○○○○○○○○○○○○○○○○○○○○○○○○○○○○○○○○○○○○○○○○○○○○○○○○○。
○○○○○○○○○○○○○○○○○○○○○○○○○○○○○○○○○○○○○○○○○○○○○○○○。
○○○○○○○○○○○○○○○○○○○○○○○○○○○○○○○○○○○○○○○○○○○○○○○○。

○○○○○○○○○○○○○○○○○○○○○○○○○○○○○○○○○○○○○○○○○○○○○○○○。
○○○○○○○○○○○○○○○○○○○○○○○○○○○○○○○○○○○○○○○○○○○○○○○○。
○○○○○○○○○○○○○○○○○○○○○○○○○○○○○○○○○○○○○○○○○○○○○○○○。
○○○○○○○○○○○○○○○○○○○○○○○○○○○○○○○○○○○○○○○○○○○○○○○○。

% 卒論本文と同様に bibtex を使っても良い。
\begin{thebibliography}{99}
\bibitem{文献01} XXXXXX, "XXXXXXX," XXXXXXXX, Vol. XX, No. XX, pp. XX-XX, XXXX.
\bibitem{文献02} XXXXXX, "XXXXXXX," XXXXXXXX, Vol. XX, No. XX, pp. XX-XX, XXXX.
\end{thebibliography}
%参考文献の書き方は、電子情報通信学会や情報処理学会の投稿のしおりに従ってください。


\end{document}
